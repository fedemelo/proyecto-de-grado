\chapter{Introducción}

\begin{resumen}
  NES es una plataforma web de la Universidad de los Andes que centraliza sus recursos y servicios. El Perfil del estudiante complementa esta plataforma al proveer información académica y socioeconómica del estudiantado, antes faltante e imprescindible para mejorar las consejerías a estudiantes y las decisiones administrativas.
\end{resumen}

\section{Contexto}

% TODO: Poner que Uniandes está a la vanguardia de la educación en Colombia
La Universidad de los Andes es una institución educativa colombiana de gran renombre, reconocida por su excelencia académica y su compromiso con la formación integral de sus estudiantes. En aras de propulsar el éxito de sus estudiantes en todo ámbito, la Universidad realiza un esfuerzo constante por ofrecer servicios y recursos que fomenten el desarrollo académico, social, personal y profesional de sus estudiantes, así como el bienestar de la comunidad universitaria en general.

\subsection{La plataforma \textit{No estás solo}}

En los últimos dos años, la Vicedecantura de Asuntos Estudiantiles de la Universidad de los Andes identificó dos dolencias significativas concernientes al estudiantado y al profesorado, respectivamente. La primera, consiste en el desconocimiento por parte de los estudiantes de la inmensa cantidad de recursos y servicios de apoyo que la Universidad pone a su disposición en pos de su éxito académico, social, personal y profesional. La segunda, radica en la dificultad y falta de recursos que tenían los profesores a la hora de ejercer su labor como consejeros en apoyo a los estudiantes.

Como herramienta para subsanar estas dolencias, la Vicedecantura ha propulsado el desarrollo y la extensión de una aplicación web: \textit{No estás solx} (en adelante, NES). Esta plataforma fue concebida por la Vicedecanatura de Asuntos Estudiantiles de la Facultad de Ingeniería como una solución integral para garantizar que todos los miembros de la comunidad, especialmente los estudiantes, puedan acceder de manera centralizada y clara a todos los recursos y servicios disponibles.

Desde la perspectiva del estudiantado, NES centraliza el acceso a todas las herramientas que fomentan su éxito académico, personal y profesional. Esto incluye información sobre eventos académicos y culturales, procesos y solicitudes administrativas, y servicios de apoyo como consejerías y tutorías. Sumado a eso, facilita el acceso de cada estudiante a su red de apoyo, compuesta por profesores consejeros, coordinadores académicos y otros profesionales de la Universidad. Toda esta información previamente se encontraba dispersa en las distintas plataformas de la Universidad, lo que dificultaba su acceso y su uso conjunto. Esto suponía un reto importante para los estudiantes, quienes debían navegar entre múltiples sistemas y portales para encontrar la información que necesitaban, además de constituir una barrera de entrada significativa para los estudiantes menos experimentados.

Para los profesores y otros usuarios administrativos, NES constituye una herramienta que facilita su labor como consejeros y mejora su capacidad de toma de decisiones. La plataforma incluye una base de preguntas frecuentes y guías para poder orientar a los estudiantes en dificultades académicas, personales o financieras. Finalmente, para los coordinadores académicos, NES contribuye a descongestionar las solicitudes frecuentes mediante herramientas de autogestión, permitiendo que los estudiantes resuelvan problemas simples por sí mismos y optimizando el uso del tiempo en las coordinaciones. 

En conjunto, NES busca mejorar la experiencia universitaria al simplificar procesos y fortalecer los canales de apoyo.

\subsection{La necesidad del Perfil del estudiante}

NES representa un avance significativo en la centralización y claridad de los recursos y servicios ofrecidos por la Universidad. Sin embargo, previo a la implementación del Perfil del estudiante, los profesores y administrativos enfrentaban desafíos al intentar ofrecer consejerías personalizadas o tomar decisiones informadas. La información sobre los estudiantes estaba dispersa en múltiples sistemas y formatos, lo que dificultaba tener una visión integral de su contexto académico y socioeconómico.

El Perfil del estudiante surge como una respuesta directa a esta necesidad. Este módulo se integra dentro de NES para ofrecer una herramienta que:
\begin{itemize}
    \item Centralice y organice la información relevante sobre los estudiantes.
    \item Presente esta información de manera intuitiva, clara y accesible, sin sacrificar la exhaustividad.
    \item Facilite a profesores y administrativos brindar consejerías personalizadas y tomar decisiones basadas en datos.
\end{itemize}

Con estas características, el Perfil del estudiante no solo mejora la experiencia de los usuarios de NES, sino que también fortalece su capacidad de impacto al abordar de manera directa las problemáticas identificadas por la Vicedecanatura de Asuntos Estudiantiles.

\section{Objetivos}

El objetivo general de este proyecto es construir el Perfil del estudiante. 

Más específicamente, el objetivo radica en proporcionar a todo el estudiantado y al profesorado de la Universidad de los Andes una herramienta digital que les permita consultar información relacionada con el desempeño académico actual y pasado, así como con el contexto socioeconómico, de los estudiantes de la Universidad.

\subsection{Objetivos específicos}

El objetivo general enunciado recién se desglosa en los siguientes objetivos específicos:
\begin{itemize}
	\item Conseguir, mediante una herramienta de software, el acceso a la información académica de los estudiantes de la Universidad de los Andes de manera que pueda interactuar con una aplicación web, idealmente siguiendo los lineamientos de un API REST.
	\item Diseñar e implementar un módulo de la aplicación web NES que extraiga y exhiba la información académica de cualquier estudiante de la Universidad de los Andes. Realizar la distinción entre la forma en la que se presenta la información para estudiantes de pregrado y estudiantes de posgrado, en particular, distinguir entre los resultados de un mismo estudiante en su pregrado y en su posgrado, teniendo completa trazabilidad de su trayectoria académica cuando sea el caso.
	\item Conectar el módulo con la navegación y funcionalidades ya existentes en la aplicación web NES, de forma que cada estudiante pueda tener acceso a su información académica (mas no a la de otros estudiantes), cada profesor consejero pueda tener acceso a la información académica de sus estudiantes aconsejados y otros usuarios específicos, determinados por la Vicedecantura, puedan tener acceso a la información académica de los estudiantes que les conciernan.
	\item Garantizar la integridad y seguridad de la información académica de los estudiantes de la Universidad de los Andes, tanto informáticamente como en su presentación, de forma que se cumpla con la normativa de protección de datos personales y de información académica de la Universidad.
\end{itemize}

\subsection{Resultados esperados}

El presente proyecto se puede dar por culminado una vez se satisfaga el siguiente conjunto de resultados:
\begin{itemize}
	\item Existencia de un módulo de la aplicación web NES que permita a cada estudiante de la Universidad de los Andes, tanto de pregrado como de posgrado, consultar información relacionada con su desempeño académico.
	\item Existencia de un módulo de la aplicación web NES que permita a cada profesor de la Universidad de los Andes consultar información relacionada con el desempeño académico de cada uno de sus estudiantes aconsejados, de forma individual.
	\item Existencia de un módulo de la aplicación web NES que permita a usuarios específicos, como decanos, directores de programas u otros directivos de la Universidad de los Andes, consultar información relacionada con el desempeño académico de los estudiantes de la Universidad que les conciernan.
\end{itemize}

\section{Usuarios y requerimientos}
Describe a los usuarios principales del sistema, sus roles y las necesidades específicas identificadas. Aquí también puedes incluir cómo se realizó la recolección y análisis de requerimientos, y un resumen de las historias de usuario que guiaron el diseño del sistema.

\section{Arquitectura general del sistema}
Introduce la arquitectura three-tier (datos, lógica y presentación). Proporciona una visión general de cómo estas capas están integradas y cómo contribuyen al funcionamiento del Perfil del estudiante.