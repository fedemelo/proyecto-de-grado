\chapter*{Agradecimientos}

Hay una infinidad de personas a quienes debo un agradecimiento en este proyecto. En este espacio, me gustaría destacar a algunas de ellas.

Antes que a nadie, quiero agradecer a Angélica, William y Sebastián: mi familia. Su amor y apoyo incondicional moldean mi vida y forjan mi persona. A ellos les debo todo lo que soy.

Luego, debo agradecer a mis amigos más cercanos, quienes son un pilar fundamental en mi vida y son los cimientos de mi felicidad. Especialmente, quiero agradecer, por un lado, a David, Santiago y Andrés; y por otro, a Laura, Laura, Mariana y Sarah. Sin esos dos conjuntos de personas, mi vida sería mucho más triste.

Por mera causalidad, debo agradecer también a todos quienes han invertido su tiempo, recursos y esfuerzos en mi educación. Eso incluye a todos quienes fueron partícipes de mi formación tanto en el Colegio San Carlos como en la Universidad de los Andes. Entre otros, debo agradecimientos especiales a Jaime Rueda, quien me enseñó a estudiar; a Nicolás Rincón, quien me forzó a estudiar; a José Bocanegra, quien me enseñó sobre desarrollo web y construcción de APIs; a Ruby Casallas y a Nicolás Cardozo, quienes me enseñaron sobre calidad de software; y a todos mis profesores de la Universidad.

Por último, quiero agradecer a quienes estuvieron directamente involucrados en este proyecto. A Mariana y a Nicolás, con quienes tomamos el desarrollo de No Estas Solo desde hace más de un año. A Manuel y (nuevamente) a Santiago, quienes hicieron posible el pipeline de analítica. A Oscar, que estaba siempre pendiente de todo. Y, por supuesto, a Marcela, que me dio la oportunidad de trabajar en este proyecto y me acompañó en todo el proceso.
