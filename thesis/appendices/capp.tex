\chapter{Contribuciones al CAPP}

Este apéndice describe las contribuciones realizadas al \gls{CAPP} durante el desarrollo de este proyecto. En particular, se detallan los recursos expuestos por la \gls{API REST} para consumo exclusivo del \gls{CAPP}, a petición de Daniel Arango Cruz y Marilyn Stephany Joven Fonseca, quienes lideran el desarrollo de aquel proyecto.

\section{Contexto}

El \gls{CAPP} es un sistema que permite a los estudiantes de pregrado de la Universidad de los Andes visualizar su progreso académico. Su propósito principal es determinar si un estudiante ha cumplido con los requisitos mínimos para graduarse de un programa académico. Para ello, el \gls{CAPP} se conecta con diversas fuentes de datos, entre ellas, con el \gls{API REST} diseñado y desarrollado en este proyecto.

\section{Recursos expuestos}

Los recursos provistos fueron diseñados específicamente para el \gls{CAPP} ante solicitud de los desarrolladores. Se solicitó que la nomenclatura para los recursos y sus atributos fuera en español, en lugar de inglés, como se había manejado en el resto de la \gls{API REST}.

A causa de eso y como forma de separación semántica, se decidió agrupar los \textit{endpoints} bajo el prefijo \texttt{/capp/}. La tabla \ref{tab:capp_api_resources} detalla los \textit{endpoints} expuestos por la \gls{API REST} para el \gls{CAPP}.

\begin{table}[H]
	\centering
	\alternatecolors
	\caption{Recursos expuestos por la API REST para el CAPP}
	\label{tab:capp_api_resources}
	\begin{tabular}{p{1.5cm}p{2.6cm}p{3.8cm}p{2cm}p{3cm}}
		\hline
		\textbf{Método HTTP} & \textbf{URL}                                                                                                & \textbf{Descripción}                                                                      & \textbf{Cuerpo de la petición} & \textbf{Respuesta}                                                              \\ \hline
		\texttt{GET}         & \texttt{/capp}\newline\texttt{/materias}\newline\texttt{/estudiante}\newline\texttt{/\{codigo estudiante\}} & Obtiene todas las materias aprobadas por un estudiante en sus estudios de pregrado.       & --                             & Lista de objetos \texttt{Materia} con las materias aprobadas por el estudiante. \\
		\texttt{GET}         & \texttt{/capp}\newline\texttt{/materia}\newline\texttt{/\{codigo materia\}}                                 & Obtiene la información básica de una materia a partir de su código.                       & --                             & Objeto \texttt{Materia} correspondiente al código.                              \\
		\texttt{PUT}         & \texttt{/capp}\newline\texttt{/materias/}                                                                   & Obtiene la información básica de múltiples materias a partir de una lista de sus códigos. & Lista de códigos de materias   & Lista de objetos \texttt{Materia} correspondientes a los códigos provistos.     \\
		\texttt{GET}         & \texttt{/capp}\newline\texttt{/estudiante}\newline\texttt{/programa/}                                       & Obtiene la información de los programas académicos cursados por el estudiante.            & --                             & Lista de nombres de los programas académicos, ordenados desde el principal.     \\
	\end{tabular}
\end{table}

El recurso \verb|Materia| mencionado en la tabla es una representación simplificada de una materia, que contiene la información mínima necesaria para el \gls{CAPP}. La figura \ref{lst:subject_capp} muestra un ejemplo de esta representación. No se hace uso de la representación \verb|SubjectResponse| mencionada en la sección \ref{sec:subject_response}, % TODO: Referenciar la sección
ya que esta contiene información adicional y se incurriría en sobreexposición de datos.

\begin{figure}[h]
	\centering
	\begin{minted}{json}
{
  "codigo_curso": "MATE",
  "numero_curso": "1104",
  "numero_creditos": 3,
  "nombre_curso": "TEORIA DE GRAFOS",
  "atributo_sec": "",
  "periodo": 202320
}
  \end{minted}
	\caption{Ejemplo de representación del recurso \texttt{Materia} para el \gls{CAPP}}
	\label{lst:subject_capp}
\end{figure}
